\section{GUI}
	This section describes the design and implementation of Graphic User Interface:
\subsection{Functional Requirements}
	The GUI contains following tabs:
	\begin{itemize}
		\item \textbf{Enrollment} \\
			A new user can be dynamically add to this system.
			when enrollment starts, the user is prompted to input
			his identifications, e.g, name, age, sex, a photo would
			be better. Experienced users may choose to update their
			info in the user list.

			Next the user needs to provide a piece of utterance for
			the enrollment and training process.

			There are two ways to enroll a user:
			\begin{itemize}
				\item \textbf{Enroll by speaking}
					There is no limit of the content of the utterance, while
					it is highly recommended that the user speaks long enough
					to provide sufficient message for the enrollment.

				\item \textbf{Enroll by pre-recorded voice}
					User can upload a pre-recorded voice of a speaker. The system
					accepts the voice given and the enrollment of a speaker is done.
			\end{itemize}

			The user can train, dump or load his/her voice features after enrollment.

		\item \textbf{Recognition of a user} \\
			A enrolled user present or record a piece of utterance,
			the system tells who the person is and show user's avatar.
			Recognition of multiple pre-recorded files can be done as well.

		\item \textbf{Conversation Recognition mode} \\
			When the system turn into Conversation Recognition mode,
			it will continuously collect voice data, and determine
			who is speaking right now. Current speaker's anvatar will show up
			in screen; otherwise the name will be shown. The conversation
			audio can be downloaded and saved.
			There are some ways to visualize the speaker-distribution in the
			conversation.
			\begin{itemize}
				\item \textbf{Conversation log}
					A detailed log, including start time, stop time,
					current speaker of each period is generated.
				\item \textbf{Conversation flow graph}
					A timeline of the conversation is showed by a number of
					talking-clouds joining together, with start time, stop time
					and users' avatars labeled. Different users are presented
					with different colors.The timeline will flow to the left dynamically
					just as time elapses. The visualization of the conversation is done
					in this way.
			\end{itemize}

	\end{itemize}

\subsection{Non-Functional Requirements}
	\begin{itemize}
		\item \textbf{Enrollment Efficiency} \\
			The enrollment procedure involves training GMM or RBM model, which
			is a time consuming process. A continuous flow of user enrollment
			should not be interupted.
		\item \textbf{Platform-Independency} \\
			As we program using Python, our program should run on
			platform that supports Python.
		\item \textbf{Recognition Accuracy} \\
			The performance of the system should be good enough to
			make this system could be carried out in practical use.
	\end{itemize}
